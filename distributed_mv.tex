\begin{frame}[fragile]
\frametitle{Compute homology via subcomplexes}
\begin{figure}
\begin{tikzpicture}[scale=.5, y=0.6pt, x=.75pt]
     %row 1
%     \node[font=\large] at (-50, 800) {$(K, U)$};
     \draw[fill=afragreen, draw = black,  line width=2]  (47,800) circle (10pt);  
     \draw[fill=afragreen, draw = black, line width=2]  (117,800) circle (10pt);   
     \draw[fill=afragreen, draw = black, line width=2]  (190,800) circle (10pt); 
     \draw[fill=afragreen, draw = black, line width=2]  (261,800) circle (10pt); 
     \begin{pgfonlayer}{edges}
            \path[draw=black,fill=black,line width=2] (47,800) -- (117, 800);
            \path[draw=black,fill=black,line width=2] (117,800) -- (190, 800);
            \path[draw=black,fill=black,line width=2] (190,800) -- (261, 800);
      \end{pgfonlayer}      
      \draw[draw, color=afrablue, fill=none, line join=round,draw opacity=0.978,line width=2] (117, 800) ellipse (100 and 45) node[anchor=north, xshift=-.55in] {$U_0$};
      \draw[draw, color=afrapurple, fill=none, line join=round,draw opacity=0.978,line width=2] (190, 800) ellipse (100 and 45) node[anchor=north, xshift=.6in] {$U_1$};
    \end{tikzpicture}
\caption{Space and Cover}
\label{fig:space-n-cover}
\end{figure}
When we have a pair of sets there is another short exact sequence:
\begin{tikzcd}
 C_*(U_0 \cap U_1) \arrow{r}{x \mapsto (x,x)} & C_*(U_0) \bigoplus C_*(U_1) \arrow{r}{ (x,y) \mapsto x-y} & C_*(U_0 \bigcup U_1) = C_*(X) \\
 \end{tikzcd}
When you have intersections of 3 or more sets, this becomes the mayer vietoris spectral sequence, with the following initial data:
\[ E^0_{p,q} = \langle p\text{-chains in a }q\textrm{-way intersection} \rangle \]
\end{frame}

\begin{frame}[fragile]
\frametitle{Mayer Vietoris Spectral Sequence}
\begin{figure}[h]
\centering
\begin{tikzcd}[scale=1,
execute at end scope={
\only<3>{
\begin{pgfonlayer}{edges}
\node[xshift=-1em,yshift=.5em] (c2) at (e20.north west) {\footnotesize$C_2\left(K^{\C}\right)$};
\node[xshift=-1em,yshift=.5em] (c1) at (e10.north west) {\footnotesize$C_1\left(K^{\C}\right)$};
\node[xshift=-1em,yshift=.5em] (c0) at (e00.north west) {\footnotesize$C_0\left(K^{\C}\right)$};
\draw[opacity=.5,line width=7mm,line cap=round,color=afrablue] (e20.center) to (e02.center); 
\draw[opacity=.5,line width=7mm,line cap=round,color=afragreen] (e10.center) to (e01.center); 
\draw[opacity=.5,line width=7mm,line cap=round,color=afrapurple] (e00.center) to (e00.center);
\end{pgfonlayer} 
}}]
|[alias=e30]|  \vdots \arrow{d}{\partial_{K}}& |[alias=e31]| \vdots \arrow{d}{\partial_{K}}& |[alias=e32]| \vdots \arrow{d}{\partial_{K}} \\
|[alias=e20]|E^0_{2,0}  \arrow{d}{\partial_{K}}                     &  |[alias=e21]| E^0_{2,1} \arrow{l}{\partial_{N}} \arrow{d}{\partial_{K}}&|[alias=e22]|  E^0_{2,2} \arrow{l}{\partial_{N}}  \arrow{d}{\partial_{K}}& \ldots \arrow{l}{\partial_{N}} \\ 
|[alias=e10]|E^0_{1,0} \arrow{d}{\partial_{K}}                     & |[alias=e11]| E^0_{1,1} \arrow{l}{\partial_{N}}  \arrow{d}{\partial_{K}}& |[alias=e12]| E^0_{1,2} \arrow{l}{\partial_{N}} \arrow{d}{\partial_{K}}& \ldots \arrow{l}{\partial_{N}} \\
|[alias=e00]|E^0_{0,0}                                      & |[alias=e01]| E^0_{0,1} \arrow{l}{\partial_{N}}                & |[alias=e02]| E^0_{0,2} \arrow{l}{\partial_{N}}                   & \ldots \arrow{l}{\partial_{N}} 
\end{tikzcd}
\end{figure}
\[ E^0_{p,q} = \langle p\text{-chains in a }q\textrm{-way intersection} \rangle \]
\pause
We can construct a chain complex where \[ C_d = \bigoplus_{p+q=d} E^0_{p,q} \] This is called the \emph{total} complex or \emph{blowup} complex.
\end{frame}

\begin{frame}[fragile]
\frametitle{The blowup complex}
\begin{minipage}{.35\textwidth}
\begin{tikzpicture}[scale=.5, y=0.6pt, x=.75pt]
     %row 1
%     \node[font=\large] at (-50, 800) {$(K, U)$};
     \draw[fill=afragreen, draw = black,  line width=2]  (47,800) circle (10pt);  
     \draw[fill=afragreen, draw = black, line width=2]  (117,800) circle (10pt);   
     \draw[fill=afragreen, draw = black, line width=2]  (190,800) circle (10pt); 
     \draw[fill=afragreen, draw = black, line width=2]  (261,800) circle (10pt); 
     \begin{pgfonlayer}{edges}
            \path[draw=black,fill=black,line width=2] (47,800) -- (117, 800);
            \path[draw=black,fill=black,line width=2] (117,800) -- (190, 800);
            \path[draw=black,fill=black,line width=2] (190,800) -- (261, 800);
      \end{pgfonlayer}      
      \draw[draw, color=afrablue, fill=none, line join=round,draw opacity=0.978,line width=2] (117, 800) ellipse (100 and 45);
      \draw[draw, color=afrapurple, fill=none, line join=round,draw opacity=0.978,line width=2] (190, 800) ellipse (100 and 45);
    \end{tikzpicture}
Space and Cover
\end{minipage}
\begin{minipage}{.2\textwidth}
\begin{tikzpicture}[y=1.0pt, x=0.8pt, yscale=-1, scale=.5]
      \path[draw=black,fill=afragreen, line width=3pt] (2.57,14.56) -- (1.9450, 33.7380) -- (63.8200,33.5230) -- (63.8200,47.7970) -- (107.5700,25.4260) -- (64.4450,3.0550) -- (63.8200,18.8240) -- (2.5700,14.5620);
\end{tikzpicture}
\vspace*{1cm}
\end{minipage}
\begin{minipage}{.4\textwidth}
\begin{tikzpicture}[scale=.5, y=0.6pt, x=.75pt]
    %disj 0
    \begin{scope}[shift={(143,80)}]
   % \node[font=\large] at (300, 730) {$K^0 \times \Delta^{0}$};
     \draw[draw, color=afrablue, fill=none, line join=round,draw opacity=0.978,line width=2] (117, 730) ellipse (100 and 45);
     \draw[fill=afrablue, draw = black, line width=2]  (47,730) circle (10pt);  
     \draw[fill=afrablue, draw = black, line width=2]  (117,730) circle (10pt);   
     \draw[fill=afrablue, draw = black, line width=2]  (190,730) circle (10pt); 
     \begin{pgfonlayer}{edges}
            \path[draw=black,fill=black,line width=2] (47,730) -- (117, 730);
            \path[draw=black,fill=black,line width=2] (117,730) -- (190, 730);
      \end{pgfonlayer}   
      \end{scope}
      %disj  1
       \begin{scope}[shift={(0,0)}]
   %      \node[font=\large] at (380, 680) {$K^1 \times \Delta^{1}$};
       \draw[draw, color=afrapurple, fill=none, line join=round,draw opacity=0.978,line width=2] (190, 680) ellipse (100 and 45);
     \draw[fill=afrapurple, draw = black, line width=2]  (117,680) circle (10pt);   
     \draw[fill=afrapurple, draw = black, line width=2]  (190,680) circle (10pt); 
      \draw[fill=afrapurple, draw = black, line width=2]  (261,680) circle (10pt); 
         \begin{pgfonlayer}{edges}
            \path[draw=black,fill=black,line width=2] (117,680) -- (190, 680);
            \path[draw=black,fill=black,line width=2] (190,680) -- (261, 680);
      \end{pgfonlayer}   
            \end{scope}
      %blowup edges 
      \begin{scope}[shift={(143,80)}]
     \begin{pgfonlayer}{edges}
            \path[draw=black,fill=black,line width=2] (47,730) -- (47, 600);
            \path[draw=black,fill=black,line width=2] (117,730) -- (117, 600);
      \end{pgfonlayer}     
%          \node[font=\large] at (280, 665) {$K^{[1]} \times \Delta^{[1]}$};
        \begin{pgfonlayer}{quadcell}
      \draw [fill=afragreen,  preaction={fill, afragreen}, pattern=north west lines, pattern color=black] (47, 600) rectangle (117, 730);
	\end{pgfonlayer}
	\end{scope}
\end{tikzpicture}
The blowup complex.
\end{minipage}
{\color{blue}{Definition:}} \[ K^{U} = \bigcup_{\emptyset \neq J \subset [|U|]} U_J \times \Delta^J \]
where $U_J = \bigcap_{j \in J} U_j$
\pause
\begin{theorem}[Segal]
For any closed cover $U$, $K^U$ and $K$ are homotopy equivalent.
\end{theorem}
\end{frame}


\begin{frame}
\only<3->{
\begin{figure}
\begin{minipage}[t]{.35\textwidth}
\begin{tikzpicture}[scale=.5, y=0.6pt, x=.75pt]
    %disj 0
    \begin{scope}[shift={(143,80)}]
   % \node[font=\large] at (300, 730) {$K^0 \times \Delta^{0}$};
     \draw[draw, color=afrablue, fill=none, line join=round,draw opacity=0.978,line width=2] (117, 730) ellipse (100 and 45);
     \draw[fill=afrablue, draw = black, line width=2]  (47,730) circle (10pt);  
     \draw[fill=afrablue, draw = black, line width=2]  (117,730) circle (10pt);   
     \draw[fill=afrablue, draw = black, line width=2]  (190,730) circle (10pt); 
     \begin{pgfonlayer}{edges}
            \path[draw=black,fill=black,line width=2] (47,730) -- (117, 730);
            \path[draw=black,fill=black,line width=2] (117,730) -- (190, 730);
      \end{pgfonlayer}   
      \end{scope}
      %disj  1
       \begin{scope}[shift={(0,0)}]
   %      \node[font=\large] at (380, 680) {$K^1 \times \Delta^{1}$};
       \draw[draw, color=afrapurple, fill=none, line join=round,draw opacity=0.978,line width=2] (190, 680) ellipse (100 and 45);
     \draw[fill=afrapurple, draw = black, line width=2]  (117,680) circle (10pt);   
     \draw[fill=afrapurple, draw = black, line width=2]  (190,680) circle (10pt); 
      \draw[fill=afrapurple, draw = black, line width=2]  (261,680) circle (10pt); 
         \begin{pgfonlayer}{edges}
            \path[draw=black,fill=black,line width=2] (117,680) -- (190, 680);
            \path[draw=black,fill=black,line width=2] (190,680) -- (261, 680);
      \end{pgfonlayer}   
            \end{scope}
\end{tikzpicture}
\vspace*{1cm}
\end{minipage}
\begin{minipage}{.2\textwidth}
\begin{tikzpicture}[y=1.0pt, x=0.8pt, yscale=-1, scale=.5]
      \path[draw=black,fill=afragreen, line width=3pt] (2.57,14.56) -- (1.9450, 33.7380) -- (63.8200,33.5230) -- (63.8200,47.7970) -- (107.5700,25.4260) -- (64.4450,3.0550) -- (63.8200,18.8240) -- (2.5700,14.5620);
\end{tikzpicture}
\vspace*{1cm}
\end{minipage}
\begin{minipage}[t]{.4\textwidth}
\begin{tikzpicture}[scale=.5, y=0.6pt, x=.75pt]
    %disj 0
    \begin{scope}[shift={(143,80)}]
   % \node[font=\large] at (300, 730) {$K^0 \times \Delta^{0}$};
     \draw[draw, color=afrablue, fill=none, line join=round,draw opacity=0.978,line width=2] (117, 730) ellipse (100 and 45);
     \draw[fill=afrablue, draw = black, line width=2]  (47,730) circle (10pt);  
     \draw[fill=afrablue, draw = black, line width=2]  (117,730) circle (10pt);   
     \draw[fill=afrablue, draw = black, line width=2]  (190,730) circle (10pt); 
     \begin{pgfonlayer}{edges}
            \path[draw=black,fill=black,line width=2] (47,730) -- (117, 730);
            \path[draw=black,fill=black,line width=2] (117,730) -- (190, 730);
      \end{pgfonlayer}   
      \end{scope}
      %disj  1
       \begin{scope}[shift={(0,0)}]
   %      \node[font=\large] at (380, 680) {$K^1 \times \Delta^{1}$};
       \draw[draw, color=afrapurple, fill=none, line join=round,draw opacity=0.978,line width=2] (190, 680) ellipse (100 and 45);
     \draw[fill=afrapurple, draw = black, line width=2]  (117,680) circle (10pt);   
     \draw[fill=afrapurple, draw = black, line width=2]  (190,680) circle (10pt); 
      \draw[fill=afrapurple, draw = black, line width=2]  (261,680) circle (10pt); 
         \begin{pgfonlayer}{edges}
            \path[draw=black,fill=black,line width=2] (117,680) -- (190, 680);
            \path[draw=black,fill=black,line width=2] (190,680) -- (261, 680);
      \end{pgfonlayer}   
            \end{scope}
      %blowup edges 
      \begin{scope}[shift={(143,80)}]
     \begin{pgfonlayer}{edges}
            \path[draw=black,fill=black,line width=2] (47,730) -- (47, 600);
            \path[draw=black,fill=black,line width=2] (117,730) -- (117, 600);
      \end{pgfonlayer}     
%          \node[font=\large] at (280, 665) {$K^{[1]} \times \Delta^{[1]}$};
        \begin{pgfonlayer}{quadcell}
      \draw [fill=afragreen,  preaction={fill, afragreen}, pattern=north west lines, pattern color=black] (47, 600) rectangle (117, 730);
	\end{pgfonlayer}
	\end{scope}
\end{tikzpicture}
\end{minipage}
\end{figure}
}
\only<4>{
\begin{figure}
\begin{minipage}[t]{.35\textwidth}
\begin{tikzpicture}[scale=.5, y=0.6pt, x=.75pt]
     %row 1
%     \node[font=\large] at (-50, 800) {$(K, U)$};
     \draw[fill=afragreen, draw = black,  line width=2]  (47,800) circle (10pt);  
     \draw[fill=afragreen, draw = black, line width=2]  (117,800) circle (10pt);   
     \draw[fill=afragreen, draw = black, line width=2]  (190,800) circle (10pt); 
     \draw[fill=afragreen, draw = black, line width=2]  (261,800) circle (10pt); 
      \draw[draw, color=afrablue, fill=none, line join=round,draw opacity=0.978,line width=2] (117, 800) ellipse (100 and 45);
      \draw[draw, color=afrapurple, fill=none, line join=round,draw opacity=0.978,line width=2] (190, 800) ellipse (100 and 45);
    \end{tikzpicture}
\end{minipage}
\begin{minipage}{.2\textwidth}
\begin{tikzpicture}[y=1.0pt, x=0.8pt, yscale=-1, scale=.5]
      \path[draw=black,fill=afragreen, line width=3pt] (2.57,14.56) -- (1.9450, 33.7380) -- (63.8200,33.5230) -- (63.8200,47.7970) -- (107.5700,25.4260) -- (64.4450,3.0550) -- (63.8200,18.8240) -- (2.5700,14.5620);
\end{tikzpicture}
\vspace*{1cm}
\end{minipage}
\begin{minipage}[t]{.35\textwidth}
\begin{tikzpicture}[scale=.5, y=0.6pt, x=.75pt]
     %row 1
%     \node[font=\large] at (-50, 800) {$(K, U)$};
     \draw[fill=afragreen, draw = black,  line width=2]  (47,800) circle (10pt);  
     \draw[fill=afragreen, draw = black, line width=2]  (117,800) circle (10pt);   
     \draw[fill=afragreen, draw = black, line width=2]  (190,800) circle (10pt); 
     \draw[fill=afragreen, draw = black, line width=2]  (261,800) circle (10pt); 
     \begin{pgfonlayer}{edges}
      \path[draw=black,fill=black,line width=2] (47,800) -- (117, 800);
       \path[draw=black,fill=black,line width=2] (117,800) -- (190, 800);
      \path[draw=black,fill=black,line width=2] (190,800) -- (261, 800);
      \end{pgfonlayer}      
      \draw[draw, color=afrablue, fill=none, line join=round,draw opacity=0.978,line width=2] (117, 800) ellipse (100 and 45);
      \draw[draw, color=afrapurple, fill=none, line join=round,draw opacity=0.978,line width=2] (190, 800) ellipse (100 and 45);
    \end{tikzpicture}
\end{minipage}
\end{figure}
}
\only<5->{
\vspace{-1cm}
\begin{figure}
\begin{minipage}[t]{.35\textwidth}
\begin{tikzpicture}[scale=.5, y=0.6pt, x=.75pt]
    %disj 0
    \begin{scope}[shift={(143,80)}]
   % \node[font=\large] at (300, 730) {$K^0 \times \Delta^{0}$};
     \draw[draw, color=afrablue, fill=none, line join=round,draw opacity=0.978,line width=2] (117, 730) ellipse (100 and 45);
     \draw[fill=afrablue, draw = black, line width=2]  (47,730) circle (10pt);  
     \draw[fill=afrablue, draw = black, line width=2]  (117,730) circle (10pt);   
     \draw[fill=afrablue, draw = black, line width=2]  (190,730) circle (10pt); 
   
      \end{scope}
      %disj  1
       \begin{scope}[shift={(0,0)}]
   %      \node[font=\large] at (380, 680) {$K^1 \times \Delta^{1}$};
       \draw[draw, color=afrapurple, fill=none, line join=round,draw opacity=0.978,line width=2] (190, 680) ellipse (100 and 45);
     \draw[fill=afrapurple, draw = black, line width=2]  (117,680) circle (10pt);   
     \draw[fill=afrapurple, draw = black, line width=2]  (190,680) circle (10pt); 
      \draw[fill=afrapurple, draw = black, line width=2]  (261,680) circle (10pt);  
            \end{scope}
      %blowup edges 
      \begin{scope}[shift={(143,80)}]
     \begin{pgfonlayer}{edges}
            \path[draw=black,fill=black,line width=2] (47,730) -- (47, 600);
            \path[draw=black,fill=black,line width=2] (117,730) -- (117, 600);
      \end{pgfonlayer}     
	\end{scope}
\end{tikzpicture}
\end{minipage}
\begin{minipage}{.2\textwidth}
\begin{tikzpicture}[y=1.0pt, x=0.8pt, yscale=-1, scale=.5]
      \path[draw=black,fill=afragreen, line width=3pt] (2.57,14.56) -- (1.9450, 33.7380) -- (63.8200,33.5230) -- (63.8200,47.7970) -- (107.5700,25.4260) -- (64.4450,3.0550) -- (63.8200,18.8240) -- (2.5700,14.5620);
\end{tikzpicture}
\vspace*{1cm}
\end{minipage}
\begin{minipage}[t]{.4\textwidth}
\begin{tikzpicture}[scale=.5, y=0.6pt, x=.75pt]
    %disj 0
    \begin{scope}[shift={(143,80)}]
   % \node[font=\large] at (300, 730) {$K^0 \times \Delta^{0}$};
     \draw[draw, color=afrablue, fill=none, line join=round,draw opacity=0.978,line width=2] (117, 730) ellipse (100 and 45);
     \draw[fill=afrablue, draw = black, line width=2]  (47,730) circle (10pt);  
     \draw[fill=afrablue, draw = black, line width=2]  (117,730) circle (10pt);   
     \draw[fill=afrablue, draw = black, line width=2]  (190,730) circle (10pt); 
     \begin{pgfonlayer}{edges}
            \path[draw=black,fill=black,line width=2] (47,730) -- (117, 730);
            \path[draw=black,fill=black,line width=2] (117,730) -- (190, 730);
      \end{pgfonlayer}   
      \end{scope}
      %disj  1
       \begin{scope}[shift={(0,0)}]
   %      \node[font=\large] at (380, 680) {$K^1 \times \Delta^{1}$};
       \draw[draw, color=afrapurple, fill=none, line join=round,draw opacity=0.978,line width=2] (190, 680) ellipse (100 and 45);
     \draw[fill=afrapurple, draw = black, line width=2]  (117,680) circle (10pt);   
     \draw[fill=afrapurple, draw = black, line width=2]  (190,680) circle (10pt); 
      \draw[fill=afrapurple, draw = black, line width=2]  (261,680) circle (10pt); 
         \begin{pgfonlayer}{edges}
            \path[draw=black,fill=black,line width=2] (117,680) -- (190, 680);
            \path[draw=black,fill=black,line width=2] (190,680) -- (261, 680);
      \end{pgfonlayer}   
            \end{scope}
      %blowup edges 
      \begin{scope}[shift={(143,80)}]
     \begin{pgfonlayer}{edges}
            \path[draw=black,fill=black,line width=2] (47,730) -- (47, 600);
            \path[draw=black,fill=black,line width=2] (117,730) -- (117, 600);
      \end{pgfonlayer}     
%          \node[font=\large] at (280, 665) {$K^{[1]} \times \Delta^{[1]}$};
        \begin{pgfonlayer}{quadcell}
      \draw [fill=afragreen,  preaction={fill, afragreen}, pattern=north west lines, pattern color=black] (47, 600) rectangle (117, 730);
	\end{pgfonlayer}
	\end{scope}
\end{tikzpicture}
\end{minipage}
\end{figure}
}
\vspace{-1cm}
\begin{itemize}
\item<2-3>Obvious parallelism\only<3>{: from this two step filtration}
\only<3>{
\begin{enumerate}
\item<3>{As long as one finds as ``balanced" cover with ``small" intersections then one can achieve near-ideal parallelism   [L., Zomorodian, '14]}
\end{enumerate}
}
\item[Issue:]<4-> Top filtration unrelated to a possible input filtration.
\item[Issue:]<5-> Bottom filtration loses parallelism.
\item[Solution]<6-> Reduce bottom filtration order, using the top filtration order. 
\end{itemize}
\only<7->{\textbf{This works, right?}} \only<8->{\textbf{Yes:} there exists a spectral sequence...}
\end{frame}



\begin{frame}
%\frametitle{main result}
%\textbf{Given} A filtration $F$ on a complex $K$ of size $m$ and cover with $p$ sets such that the resulting blowup complex has size $m+n$. \\
%\pause
%\vspace{1in}
%\textbf{Then} After reducing the disjoint union we can compute the persistent homology of $F$ using $O\left(mn^2\right)$  time and uses only $O(mn)$ space. \\
%\vspace{1in}
%\pause
%\textbf{Now:} Sketch the algorithm.
%\end{frame}
%
%\begin{frame}
%\frametitle{distributed mayer vietoris}
%\[  \left(%
%  \vcenter\bgroup\hbox\bgroup
%  \tikzpicture[
%    x=.75\baselineskip,
%    y=.75\baselineskip,
%  ]
%    \mblock[afragreen](0,1)\partial_{1}(3,3)
%     \mblock[afrablue](3,-2)\partial_{2}(3,3)
%     \mblock[afrapurple](6,-5)\partial_{3}(3,3)
%     \mblock[afragreen](9,1)(1,3)
%     \mblock[afrablue](9,-2)(1,3)
%     \mblock[afrablue](10,-2)(1,3)
%     \mblock[afrapurple](10,-5)(1,3)
%      \mblock[afrapurple](11,-5)(1,3)
%      \mblock[afragreen](11,1)(1,3)
%      \mblock[afragreen](12,1)(1,3)
%      \mblock[afrablue](12,-2)(1,3)
%       \mblock[afrapurple](12,-5)(1,3)
%       \mblock[gray](9,-9)(4,4)
%  \endtikzpicture
%  \egroup
%  \egroup \right) \]
%  \begin{enumerate}
%  \item[Step 1]<1-> Reduce all blocks of a fixed color independently. $\partial_iV_i \rightarrow R_i$
%  \item[Step 2]<2-> Row reduce disjoint union (and update other blocks) $S^{-1}[R_i \mid L_i] \rightarrow [P_i \mid \tilde{L}_i]$ $P_i$ has at most 1 nz per column.
%  \item[Step 3]<3-> Permute columns and rows into correct order $\pi_F[P_i \mid \tilde{L}_i] = T_i$
%  \item[Step 4]<4-> Do final reduction \emph{carefully} $T_i\tilde{V}_i = \tilde{R_i}$
%  \end{enumerate}
\end{frame}
