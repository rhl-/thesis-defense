%\usepackage[all]{xy}
\usepackage{graphicx,dsfont, amsmath,amssymb}

\usepackage[absolute,overlay]{textpos}
%\usepackage{subfigure}
%\usepackage{subcaption}
\usepackage{rotating}
\usepackage{cases}
\usepackage{tikz}
\usepackage[document]{ragged2e}
\usepackage[labelformat=empty]{caption}
\usepackage{pgfplots}
\usepackage{ifthen}
\usepackage{etoolbox} % for \ifdefstring
\usetikzlibrary{patterns}
\usetikzlibrary{matrix}
\usetikzlibrary{positioning}
\usepackage{tikz-cd}

\usepackage{animate}
\usepackage{ifthen}
\usepackage{animate}
\usepgfplotslibrary{external}
\usepackage{svg}
\usepackage{amsmath}
\usepackage{tikz-cd}
\usetikzlibrary{3d,calc,intersections,patterns,fadings}
\usepackage{pgfplotstable}
\usepackage{filecontents}
\usepackage{lmodern}
\usetikzlibrary{fit, shadows, arrows, positioning, external, matrix}

%\tikzexternalize[shell escape=-enable-write18, mode=list and make, prefix=tikz/]

%% Colors

\colorlet{lightgray}{black!15!white}
\colorlet{darkgreen}{green!75!black}
\colorlet{darkblue}{blue!75!black}
\colorlet{darkred}{red!75!black}
\xdefinecolor{ritorange}{rgb}{1, .37, 0}
\xdefinecolor{white}{rgb}{255,255,255}
\xdefinecolor{ritbrown}{cmyk}{0,60,70,85}
\xdefinecolor{gold}{rgb}{0.78, 0.68, 0.37}
\definecolor{lightgray}{gray}{.8}
%%%Beamer Theme Color Settings
%\setbeamercolor{palette primary}{fg = white, bg = ritbrown!70!black}
%\setbeamercolor{palette secondary}{fg = black, bg = ritbrown!70!white}
%\setbeamercolor{palette tertiary}{fg = white, bg = ritbrown}
%\setbeamercolor{palette quaternary}{fg = black, bg = afra}
%\setbeamercolor{titlelike}{fg = black, bg = afra}
%\setbeamercolor{date in head/font}{fg = black, bg = afra}
%\setbeamercolor{frametitle}{fg = black, bg = afra}
%\setbeamercolor{separation line}{fg = afra}
%\setbeamertemplate{navigation symbols}{}
%\setbeamercolor{frame title}{use=structure,fg=white,bg=ritbrown!75!black}
%\setbeamercolor{frame title alerted}{use=alerted text,fg=white,bg=alerted text.fg!75!black}
%\setbeamercolor{frame title example}{use=example text,fg=white,bg=example text.fg!75!black}
%\setbeamercolor{frame body}{parent=normal text,use=frame title,bg=frame title.bg!10!bg}
%\setbeamercolor{frame body alerted}{parent=normal text,use=frame title alerted,bg=frame title alerted.bg!10!bg}
%\setbeamercolor{frame body example}{parent=normal text,use=frame title example,bg=frame title example.bg!10!bg}
%\setbeamercolor{alerted text}{fg = ritbrown!75!black}
%\setbeamercolor{example text}{fg = afra!70!gold!90!white}
%\setbeamercolor{normal text}{fg = black, bg = white!25!white}
%\setbeamercolor{sidebar}{fg = white, bg = ritbrown!70!black}
%\setbeamercolor{section in sidebar}{fg = white, bg = ritbrown!70!black}
%\setbeamercolor{section in sidebar current}{fg = white, bg = ritbrown!70!black}
%\setbeamercolor{subsection in sidebar}{fg = white, bg = ritbrown!70!black}
%\setbeamercolor{subsection in sidebar current}{fg = white, bg = ritbrown!70!black}
%\setbeamercolor{palette sidebar tertiary}{fg=white!80!ritbrown}
%\setbeamercolor{palette sidebar quaternary}{fg=white}
%%
%%%%%%%%%%
%Afra's Colors
%%%%%%%%%
\definecolor{afra}{RGB}{198,168,208}
\definecolor{afrablue}{RGB}{143,166,215}
\definecolor{afragreen}{RGB}{182,215,112}
\definecolor{afrapurple}{RGB}{218,177,239}
\definecolor{darkgray}{gray}{0.3}
\definecolor{afrapurplelight}{RGB}{198,168,208}
\definecolor{afrabluedark}{RGB}{75,113,191}
\definecolor{afrabluelight}{RGB}{143,166,215}
\definecolor{afragreendark}{RGB}{154,191,75}
\definecolor{afragreenlight}{RGB}{182,215,112}
\definecolor{afrapurpledark}{RGB}{162,75,191}

%\setbeamertemplate{footline}[frame number]
% \usepackage{beamerthemesplit} // Activate for custom appearance
%\usepackage{mathspec}
%\setallmainfonts(Digits,Latin){rhl}
%\setmainfont{rhl}
%\setsansfont{rhl}

\makeatletter
\tikzset{% 
    complex_triangle/.style={%
        fill=green
    }
}
\tikzset{
  beamer externalizing/.style={%
    execute at end picture={%
      \tikzifexternalizing{%
        \ifbeamer@anotherslide
        \pgfexternalstorecommand{\string\global\string\beamer@anotherslidetrue}%
        \fi
      }{}%
    }%
  },
  external/optimize=false
}
\makeatother
\tikzset{every picture/.style={beamer externalizing}}
%\tikzexternalize[prefix=tikz/]

% defining the new dimensions and parameters
\newlength{\hatchspread}
\newlength{\hatchthickness}
\newlength{\hatchshift}
\newcommand{\hatchcolor}{}
% declaring the keys in tikz
\tikzset{hatchspread/.code={\setlength{\hatchspread}{#1}},
         hatchthickness/.code={\setlength{\hatchthickness}{#1}},
         hatchshift/.code={\setlength{\hatchshift}{#1}},% must be >= 0
         hatchcolor/.code={\renewcommand{\hatchcolor}{#1}}}
% setting the default values
\tikzset{hatchspread=3pt,
         hatchthickness=0.4pt,
         hatchshift=.1pt,% must be >= 0
         hatchcolor=gray}
% declaring the pattern
\pgfdeclarepatternformonly[\hatchspread,\hatchthickness,\hatchshift,\hatchcolor]% variables
   {custom north west lines}% name
   {\pgfqpoint{\dimexpr-2\hatchthickness}{\dimexpr-2\hatchthickness}}% lower left corner
   {\pgfqpoint{\dimexpr\hatchspread+2\hatchthickness}{\dimexpr\hatchspread+2\hatchthickness}}% upper right corner
   {\pgfqpoint{\dimexpr\hatchspread}{\dimexpr\hatchspread}}% tile size
   {% shape description
    \pgfsetlinewidth{\hatchthickness}
    \pgfpathmoveto{\pgfqpoint{0pt}{\dimexpr\hatchspread+\hatchshift}}
    \pgfpathlineto{\pgfqpoint{\dimexpr\hatchspread+0.15pt+\hatchshift}{-0.15pt}}
    \ifdim \hatchshift > 0pt
      \pgfpathmoveto{\pgfqpoint{0pt}{\hatchshift}}
      \pgfpathlineto{\pgfqpoint{\dimexpr0.15pt+\hatchshift}{-0.15pt}}
    \fi
    \pgfsetstrokecolor{\hatchcolor}
%    \pgfsetdash{{1pt}{1pt}}{0pt}% dashing cannot work correctly in all situation this way
    \pgfusepath{stroke}
   }


\newcommand{\setmath}[2]{\pgfmathparse{#2}\edef#1{\pgfmathresult}}



%% COMMANDS
\newcommand*\rot{\rotatebox{90}}
\newcommand*\rota[1]{\rotatebox{#1}}
\newcommand{\bigslant}[2]{{\raisebox{.2em}{$#1$}\left/\raisebox{-.2em}{$#2$}\right.}}
\newcommand{\norm}[1]{\left|\left|#1\right|\right|}
\newcommand{\mat}[1]{{\mbox{\bf#1}\/}}
\newcommand{\bx}{\mathbf{x}}
\newcommand{\A}{\mathbf{A}}
\newcommand{\B}{\mathbf{B}}
\newcommand{\C}{\mathbf{C}}
\newcommand{\s}{\mathbf{s}}
\renewcommand{\v}[1]{\mathbf{v}_{#1}}
\renewcommand{\k}{\mathbf{k}}
\newcommand{\bu}{\mathbf{u}}
\newcommand{\bv}{\mathbf{v}}
\newcommand{\bbZ}{\mathds{Z}}
\newcommand{\bbF}{\mathds{F}}
\newcommand{\bbR}{\mathds{R}}
\newcommand{\bbC}{\mathds{C}}
\newcommand{\bbN}{\mathds{N}}
\newcommand{\bbQ}{\mathds{Q}}
\newcommand{\bbzw}{\mathds{Z}[\omega]}
\newcommand{\bbzi}{\mathds{Z}[i]}
\newcommand{\bbV}{\mathds{V}}
\newcommand{\bbS}{\mathds{S}}
\newcommand{\bbX}{\mathds{X}}
\newcommand{\bbY}{\mathds{Y}}
\newcommand{\bbG}{\mathds{G}}
\newcommand{\bbE}{\mathds{E}}
\newcommand{\U}{U}
\newcommand\sep{\mathrm{sep}}
\newcommand\diam{\mathrm{diam}}
\newcommand{\beq}[1]{\begin{equation}\label{#1}}
\newcommand{\eeq}{\end{equation}}

% Erdos-Renyi
\newcommand{\Erdos}{Erd\H{o}s}
\newcommand{\Renyi}{R{\'e}nyi}
\newcommand{\almosta}{a.a.\@}

\renewcommand{\Im}[1]{\operatorname{Im}{#1}}
\newcommand{\Ker}[1]{\operatorname{Ker}{#1}}
%\newcommand{\rank}[1]{\operatorname{rank}{#1}}
\newcommand{\Dim}[1]{\operatorname{dim}{#1}}
\newcommand{\bd}{\ensuremath{\partial}}
\DeclareMathOperator{\im}{im}
\newcommand{\betti}{\ensuremath{\beta}}
\newcommand \iso{\cong}
\newcommand \miso{\mathcal{M}^{iso}}
\newcommand{\tensor}{\otimes}
\DeclareMathOperator{\rank}{rank}
\newcommand{\Bd}[1]{\partial_{#1}}
\newcommand{\Cl}[1]{\operatorname{Cl}{#1}}
%\newcommand{\bigslant}[2]{{\raisebox{.2em}{$#1$}\left/\raisebox{-.2em}{$#2$}\right.}}
\newcommand{\card}[1]{|#1|}
\newcommand{\mv}{Mayer-Vietoris }
\newcommand{\mvb}{\mv blowup complex }
\newcommand{\K}{K}
\renewcommand{\L}{L}
%\newcommand{\C}{U}
%\newcommand{\N}{N}
\newcommand{\factor}{\card{\K^{\C}}/\card{\K}}
\newcommand{\ratio}{\factor}
\def\compcirc {\mbox{\hspace{.05cm}}\raisebox{.04cm}{\tiny  {$\circ$ }}}

\newcommand\R{\mathbb{R}}
\newcommand\N{\mathbb{N}}


\newcommand\dmtwo[1]{\left(\begin{smallmatrix}
0 & {#1}\\
{#1} & 0
\end{smallmatrix}\right)}
\newcommand\dmthree[3]{\left(\begin{smallmatrix}
0 & {#1}&{#2}\\
{#1} & 0 & {#3}\\
{#2} &  {#3}& 0
\end{smallmatrix}\right)}

\newcommand \cfunc{{\mathfrak C}}
\newcommand \hfunc{{\mathfrak H}}
\newcommand \rfunc{{\mathfrak R}}
\newcommand \func[1]{{\mathfrak #1}}
\newcommand \fin[1]{\textcolor{gray}{#1}}

\newcommand \tfunc{{\mathfrak T}}
\newcommand \ifunc{{\mathfrak I}}

\newcommand\ob{\mathrm{ob}}
\newcommand\mor{\mathrm{Mor}}
%\newcommand\im{\operatorname{Im}}
%\newcommand \iso{\cong}
%\newcommand \miso{\mathcal{M}^{iso}}
\newcommand \mmon{\mathcal{M}^{inj}}
\newcommand \mgen{\mathcal{M}^{gen}}
\newcommand \many{\mathcal{M}}
\newcommand \catc{\mathcal{C}}
\newcommand \catp{\mathcal{P}}
\newcommand \catq{\mathcal{Q}}
\newcommand \cSets{\underline{Sets}}
\newcommand \cat[1]{\mathcal{#1}}
\newcommand \cati{\mathcal{I}}
\newcommand \catd{\mathcal{D}}
\newcommand \catr{\mathbb{E}}
%\newcommand \card[1]{\left|{#1}\right|}

\setbeamercolor{framesource}{fg=white}
\setbeamerfont{framesource}{size=\tiny}
\newcommand{\source}[1]{\begin{textblock*}{8cm}(4.7cm,7.8cm)
    \begin{beamercolorbox}[ht=0.5cm,left]{framesource}
        \usebeamerfont{framesource}\usebeamercolor[fg]{framesource} Source: {#1}
    \end{beamercolorbox}
\end{textblock*}}

\pgfdeclarelayer{ball}    % declare background layer
\pgfdeclarelayer{edge}
\pgfdeclarelayer{triangle}
\pgfdeclarelayer{edges}
\pgfdeclarelayer{quadcell}
\pgfsetlayers{quadcell,edges,ball,triangle,edge,main}  % set the order of the layers (main is the standard layer)%
%\newcommand{\betti}{\beta}

%%%%%%%%%%
%Afra's Colors
%%%%%%%%%
\definecolor{afra}{RGB}{198,168,208}
\definecolor{afrablue}{RGB}{143,166,215}
\definecolor{afragreen}{RGB}{182,215,112}
\definecolor{afrapurple}{RGB}{218,177,239}
\definecolor{darkgray}{gray}{0.3}
\definecolor{afrapurplelight}{RGB}{198,168,208}
\definecolor{afrabluedark}{RGB}{75,113,191}
\definecolor{afrabluelight}{RGB}{143,166,215}
\definecolor{afragreendark}{RGB}{154,191,75}
\definecolor{afragreenlight}{RGB}{182,215,112}
\definecolor{afrapurpledark}{RGB}{162,75,191}

\definecolor{block}{RGB}{0,162,232}

\def\Put(#1,#2)#3{\leavevmode\makebox(0,0){\put(#1,#2){#3}}}
%turn off navigation for beamer
\setbeamertemplate{footline}[frame number]
\beamertemplatenavigationsymbolsempty
\usepackage{dynblocks}
\usepackage{soul}

\graphicspath{{figs/},{figs/speedup-figs/}}
\def\data{0.921656031244/3.28223363795, -2.54040213655/-0.482973014247, -0.844751008374/-2.65770211167, 2.79598718585/-0.191595474861, -3.66795511256/-1.48652801965, 1.47370635732/3.38506423654, -2.20386869455/-3.18995662836, -3.58553422279/-0.239980212622, -0.275064784083/3.87196129769, 0.267448704456/-3.80341370381, 2.62305220064/-2.12060071897, -0.110879026412/2.99008544019, 2.32272892373/-3.09021386544, -3.3800171912/1.55496983958, -1.87999249854/-3.34828948109, 2.8462152236/2.5496042126, 2.48379991896/1.68378124895, -3.56838781889/0.19745454234, 1.88048079992/2.7440373935, 2.3653448542/-2.3997700504, -2.35083110336/-2.60964081438, 2.55815900019/1.03676078039, -0.145962195604/3.69413968264, -3.79681793724/-1.06962624475, -0.275753867578/2.50038076135, 0.675379989198/3.70262495102, -2.11181552287/3.24933505206, -0.772385300185/2.97526098017, -0.34967392067/-3.18561875698, -3.14791363183/1.79446885407, -3.50240632565/1.66020050953, -0.418559718435/3.39029290533, 3.32423407025/-1.65653685632, -2.25497056752/1.88796852498, -2.88586824635/0.271758144811, -2.28219009507/2.82101056315, -1.96584613646/2.91643712731, 3.44305473568/1.98709136968, -1.4424394005/-2.26490665729, 0.0579555037661/-2.70002831519, 2.19585750406/-1.57153112157, 1.03808507177/-2.87404794413, -3.29132420516/2.10714672311, 1.70251798947/-2.15387658056, 1.82103457722/-1.83494708916, 2.2457786604/-1.55238512234, -2.43947111169/0.656482320557, 1.00778555422/2.33945886668, 0.943477302947/-2.88992137041, 0.368386770359/3.92898493334, -0.968495768794/-3.37632067922, -0.169949388136/-3.66405300824, -2.51019285613/-2.92488223447, -3.81348497717/-0.725236977049, -3.46355443364/0.861065703837, -1.17215133873/3.76349541273, 1.84677970503/2.64636375248, -2.69960085339/-1.18298434568, 3.93672771436/-0.477903130308, -2.58584779388/1.5432592263, 1.5308343636/-3.53681995933, 3.05050478273/2.44115424329, -3.89341329337/-0.914075389447, -3.54581102603/-0.460602833138, -0.898813130552/-3.58484714119, -1.3930132729/2.22519225871, -0.250446082229/3.59512375696, 0.605856998727/2.86937451189, 1.13263429332/-3.40097697598, 3.49737628581/0.560535070357, 1.39035623098/2.57288471499, 2.21493502653/-2.13236982606, -0.892022382027/-2.87180741886, 2.81647196544/1.27094245389, -2.89709047703/1.76862483511, -3.62002993237/1.12816757332, 0.547657812574/-2.87215622824, 0.49155378996/-3.81760964294, 2.07758093047/2.22584403283, 0.563403809984/-3.37784878865, -1.6851439899/-2.38106051669, 1.29914461224/-2.47672576478, 0.260115224124/-3.82626272161, -3.00396160943/2.08827337681, 3.11327165409/0.260999083881, 3.34514143367/-0.83714810883, -0.34522256144/2.64504248932, 3.31709178431/0.408448381336, -3.0396657865/0.392047481734, 1.81453633191/-2.37962678275, -1.68627017019/-1.97902553247, -3.38655428388/1.52865470037, 2.22499738622/-1.77427546986, -1.34635519293/-2.21577594494, 2.03752284563/-2.62765905144, 0.167589998452/3.18315025125, -2.18957488835/-1.28871850379, 0.574134926733/2.93531637803, 3.2765961489/0.112496696873,-3.2/-2.2,-1.7/3.7,2/-.68}
\def\radii{0, .2, .4, .5, .6, .7, .8}
 %%annulus data

\usepackage{mathtools}
\newcommand\SmallMatrix[1]{{%
  \tiny\arraycolsep=0.5\arraycolsep\ensuremath{\begin{pmatrix}#1\end{pmatrix}}}}

\usepackage{ifthen}

% master block matrix, should use wrappers instead of calling this
% directly
% This should be called within the tikzpicture environment
% \blockmatrix
%  {width}
%  {height}
%  {text}
%  {block_line_color} (can be none --> no line)
%  {block_fill_color} (can be none --> empty fill)
%  {is_diagonal} (true --> true, otherwise --> false)
%  {diagonal_line_color} (ignored if not diagonal) (can be none --> no line)
%  {diagonal_fill_color} (ignored if not diagonal) (can be noneo --> empty fill)
%  {diagonal_offset} (half diagonal width in tikz units)
% Note: colors here are not rgb, they are defined colors
\newcommand{\blockmatrix}[9]{
  \draw[draw=#4,fill=#5] (0,0) rectangle( #1,#2);
  \ifthenelse{\equal{#6}{true}}
  {
    \draw[draw=#7,fill=#8] (0,#2) -- (#9,#2) -- ( #1,#9) -- ( #1,0) -- ( #1 - #9,0) -- (0,#2 -#9) -- cycle;
  }
  {}
  \draw ( #1/2, #2/2) node { #3};
}

% Quick implementation of a tikz right parenthesis
% \rightparen{width}
\newcommand{\rightparen}[1]{
  \begin{tikzpicture} 
    \draw (0,#1/2) arc (0:30:#1);
    \draw (0,#1/2) arc (0:-30:#1);
  \end{tikzpicture}%this comment is necessary
}

% Quick implementation of a tikz left parenthesis
% \leftparen{width}
\newcommand{\leftparen}[1]{
  \begin{tikzpicture} 
    \draw (0,#1/2) arc (180:150:#1);
    \draw (0,#1/2) arc (180:210:#1);
  \end{tikzpicture}%this comment is necessary
}

% Unframed block matrix, "m" prefix to match fbox, mbox
% \blockmatrix[r,g,b]{width}{height}{text}
\newcommand{\mblockmatrix}[4][none]{
  \begin{tikzpicture} 
  \ifthenelse{\equal{#1}{none}}
  {
    \blockmatrix{#2}{#3}{#4}{none}{none}{false}{none}{none}{0.0}
  }
  {
    \definecolor{fillcolor}{rgb}{#1}
    \blockmatrix{#2}{#3}{#4}{none}{fillcolor}{false}{none}{none}{0.0}
  }
  \end{tikzpicture}%this comment is necessary
}

% Framed block matrix
% \fblockmatrix[r,g,b]{width}{height}{text}
\newcommand{\fblockmatrix}[4][none]{
  \begin{tikzpicture} 
  \ifthenelse{\equal{#1}{none}}
  {
    \blockmatrix{#2}{#3}{#4}{black}{none}{false}{none}{none}{0.0}
  }
  {
    \definecolor{fillcolor}{rgb}{#1}
    \blockmatrix{#2}{#3}{#4}{black}{fillcolor}{false}{none}{none}{0.0}
  }
  \end{tikzpicture}%this comment is necessary
}

% Diagonal block matrix
% \dblockmatrix[r,g,b]{width}{height}{text}
\newcommand{\dblockmatrix}[4][none]{
  \begin{tikzpicture} 
  \ifthenelse{\equal{#1}{none}}
  {
    \blockmatrix{#2}{#3}{#4}{black}{none}{true}{black}{none}{0.35cm}
  }
  {
    \definecolor{fillcolor}{rgb}{#1}
    \blockmatrix{#2}{#3}{#4}{black}{none}{true}{black}{fillcolor}{0.35cm}
  }
  \end{tikzpicture}%this comment is necessary
}


% Diagonal block matrix, but exposes diagonal offset
% \diagonalblockmatrix[r,g,b]{width}{height}{text}
\newcommand{\diagonalblockmatrix}[5][none]{
  \begin{tikzpicture} 

  \ifthenelse{\equal{#1}{none}}
  {
    \blockmatrix{#2}{#3}{#4}{black}{none}{true}{black}{none}{#5}
  }
  {
    \definecolor{fillcolor}{rgb}{#1}
    \blockmatrix{#2}{#3}{#4}{black}{none}{true}{black}{fillcolor}{#5}
  }

  \end{tikzpicture}%necessary comment
}

\newcommand{\valignbox}[1]{
  \vtop{\null\hbox{#1}}% necessary comment
}

% a hack so that I don't have to worry about the number of columns or
% spaces between columns in the tabular environment
\newenvironment{blockmatrixtabular}
{% necessary comment
  \begin{tabular}{
  @{}l@{}l@{}l@{}l@{}l@{}l@{}l@{}l@{}l@{}l@{}l@{}l@{}l@{}l@{}l@{}l@{}l@{}l@{}l
  @{}l@{}l@{}l@{}l@{}l@{}l@{}l@{}l@{}l@{}l@{}l@{}l@{}l@{}l@{}l@{}l@{}l@{}l@{}l
  @{}l@{}l@{}l@{}l@{}l@{}l@{}l@{}l@{}l@{}l@{}l@{}l@{}l@{}l@{}l@{}l@{}l@{}l@{}l
  @{}
  }
}
{
  \end{tabular}%necessary comment
}
\newcommand{\greenbox}[1]{ \fblockmatrix     [0.8,1.0,0.8]{.5in}{0.37in}{#1}}
\newcommand{\bluebox}[1]{ \fblockmatrix     [0.8,0.8,1.0]{.5in}{0.37in}{#1}}

\newcommand{\ssbox[2]}{   \alt<#1>{\greenbox{#2}}{\bluebox{#2}}}


\newcommand{\bunny}{{\textsf{B}}}
\newcommand{\sphere}{\textsf{S}}
\newcommand{\clique}{\textsf{C}}
\newcommand{\multiblob}{\textsf{M}}
\newcommand{\blobs}{\multiblob}
\newcommand{\gnp}{\textsf{G}}