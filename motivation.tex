\section{Motivation}
\begin{frame}
\frametitle{Applied Topology}
\begin{minipage}{0.45\columnwidth}%
\begin{tikzpicture}[scale=.5]
    \foreach[count=\p] \x / \y in \data {
\only<4->{
\begin{pgfonlayer}{ball}
      \filldraw[draw=none,fill=gray] (0,0) circle [radius=4];
      \filldraw[draw=none,fill=white] (0,0) circle [radius=2.2];
      \end{pgfonlayer}{ball}
      }
	\node[draw, circle, scale=.25, fill=white](\p) at (\x, \y) {};
    } 
 \end{tikzpicture}
\end{minipage}%
\begin{minipage}{0.55\columnwidth}%
\begin{enumerate}
\item<1->{ Set of points in $\R^2$}
\item<2->{ Looks like an annulus.}
\end{enumerate}
\end{minipage}
\begin{enumerate}
\item[] What is this?
\item[] What does it look like ?
\end{enumerate}
\only<3->{Applied topology about recovering shape from [geometric] data}
\end{frame}

\begin{frame}
\frametitle{Data Has Shape}
\begin{minipage}{0.45\columnwidth}%
\begin{tikzpicture}[scale=.5]
\begin{pgfonlayer}{ball}
      \filldraw[draw=none,fill=gray] (0,0) circle [radius=4];
      \filldraw[draw=none,fill=white] (0,0) circle [radius=2.2];
      \end{pgfonlayer}{ball}
\foreach \x / \y in \data {
	\node[draw, circle, scale=.25, fill=white] at (\x, \y) {};
	}
\end{tikzpicture}
\end{minipage}%
\begin{minipage}{0.55\columnwidth}%
\begin{description} 
\setlength{\itemindent}{-2.5em}
\item<2->   2-dimensional 
\item<3->   Approximates annulus
\end{description}
\only<5-> {Topological features of annulus:}
\setlength{\itemindent}{-3em}
\begin{description}
\item<6-> 1 component
\item<7-> 1 loop 
\end{description}

\end{minipage}
\begin{description}
\item<8->[\textbf{Goal:}] Recover \emph{topology} of annulus from point cloud
\end{description}
\end{frame}

\begin{frame}
  \frametitle{Wherefore topology? }
  \begin{minipage}{.45\textwidth}
  \begin{tikzpicture}[scale=.5]
    \foreach[count=\p] \x / \y in \data {
    	\node[draw, circle, scale=.25, fill=white](\p) at (\x, \y) {};
	\draw[red] (1) -- (100);
	\draw[bend left, green] (1) -- (96) -- (12) -- (28) -- (66) -- (60) -- (89) -- (64) -- (100); 
    } 
 \end{tikzpicture}
  \end{minipage}
  \begin{minipage}{.65\textwidth}
    Geometry is too rigid! \\
    \noindent Trouble with distance metrics:
    \begin{description}
    \item[Unreliable] Only trust small distances
    \item[Ill motivated] The metrics in use may be arbitrarily chosen
    \end{description}
    \end{minipage}
\end{frame}

\begin{frame}
  \frametitle{Wherefore topology? }
  \begin{minipage}{.25\textwidth}
  \begin{tikzpicture}[scale=.5]
    \foreach[count=\p] \x / \y in \data {
	\begin{pgfonlayer}{ball}
        \fill[gray!50,radius= .8 cm] (\x,\y) circle{};
        \end{pgfonlayer}{ball}
	\node[draw, circle, scale=.25, fill=white](\p) at (\x, \y) {};
    } 
 \end{tikzpicture}
  \end{minipage}
  \begin{minipage}{.65\textwidth}
  \begin{description}
      \item Depend only on \emph{nearness}. 
      \item \emph{Count} qualitative features.
      \item Dimension Agnostic.
      \end{description}
    \end{minipage}
    \vfill
\end{frame}

\begin{frame}{Applications of Persistent Homology}
\begin{center}
\begin{figure}
\includegraphics[height=2cm]{persistence_and_materials}
\caption{$Si02$ in different states has different P.H. Credit: Hiraoka}
\end{figure}
\end{center}
\begin{minipage}{.43\textwidth}
\begin{figure}
\includegraphics[height=2cm]{complex_on_brains}
\caption{Neuroscience. Credit: Chung}
\end{figure}
\end{minipage}
\begin{minipage}{.43\textwidth}
\hfill
\begin{figure}
\includegraphics[height=2cm]{sensors}
\caption{Sensor Networks}
\end{figure}
\vfill
\vspace{.5cm}
\end{minipage}
\end{frame}